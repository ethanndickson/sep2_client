\chapter{Background}\label{ch:background}

\section{High-Level Architecture}
The IEEE 2030.5 protocol follows a REST API architecture, and as such, adopts a client-server model.

Transmitted between clients and servers are 'Resources', all of which are defined in a standardised schema, an XSD. 
Despite the client-server model, the IEEE 2030.5 specification purposefully does not make distinctions between clients and servers, as to avoid resources having differing behaviours on each. Rather, a server exposes resources to clients, while clients retrieve, update, create and delete resources on servers.
Servers communicate with many clients, and under a set of specified requirements, clients can communicate with multiple servers.

Being the product of existing technologies, the IEEE 2030.5 resources cover a wide range of applications. As such, the specification logically groups resources into discrete 'function sets', of which there are twenty-five. 
Device manufacturers or electric utilities implementing IEEE 2030.5 need only communicate resources from function sets relevant to the purpose of the device. 

The specification defines two methods by which clients retrieve resources from server. The default method has clients 'poll' servers for the latest versions of resources on a timed interval.
The second, more modern, and more scalable method, has clients 'subscribe' to a resource, after which they will be sent notifications containing any changes to the subscribed resource from the server, without needing to poll.

Despite this, whether a resource can be subscribed to can be further refined by the server exposing the resource.
For that reason, it is often required that clients employ both polling, and subscriptions when maintaining the latest instance of a resource. \cite{AUSDOE} 
\cite{IEEE2030.5}

\section{Protocol Design}
Resources are transmitted between clients and servers using HTTP/1.1, over TCP/IP, optionally using TLS.
As a result, the protocol employs the HTTP request methods of GET, POST, PUT and DELETE for retrieving, updating, creating and deleting resources, respectively.

The specification requires that SSL/TLS certificates be signed, and all encryption done, using ECC cipher suites, with the ability to use RSA cipher suites as a fallback. Unlike regular TLS, these certificates are not only used to verify a server's identity, but are also used to verify a client's identity, using a protocol colloquially referred to as "Mutual TLS" or mTLS.

The standardised resource schema is defined in a XSD, as all transmitted resources are represented using either XML, or EXI, with the HTTP/1.1 \texttt{Content-Type} header set to \texttt{sep+xml} or \texttt{sep-exi}, respectively.

In order to connect to servers, clients must be able to resolve hostnames to IP addresses using DNS. Similarly, the specification permits the ability for clients to discover servers on a local network using DNS-SD.

\section{Usage}
The IEEE 2030.5 function sets cover a wide range of applications and uses, aiming to support as many end-user energy devices as possible. A subset of these possible use cases are as follows:
\begin{itemize}
    \item (Smart) Electricity Meters can use the 'Metering' function set to 'exchange commodity measurement information' using 2030.5 resources. \cite{IEEE2030.5}
    \item Electric Vehicle chargers may wish to have their power usage behaviour influenced by resources belonging to the 'Demand Response and Load Control' function set.
    \item Solar Inverters may use the 'Distributed Energy Resources' function set such that their energy output into the wider grid can be controlled by the utility, as to avoid strain on the grid.
\end{itemize}
