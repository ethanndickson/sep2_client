\chapter{Design}\label{ch:design}
Given the context and background surrounding the IEEE 2030.5 protocol, we can begin examining the high-level considerations, constraints, and assumptions we make in designing our implementation.

\section{Considerations}

\subsection{Modularity}
IEEE 2030.5 is a specification with a wide range of applications. It aims to provide functionality for coordinating virtually all types of end user energy devices. Under CSIP and CSIP-AUS, it can be deployed in different contexts (See Section 4.1), and can require software on both the side of the electric utility, and the end-user energy device.

For this reason, instead of developing any single binary application, we've instead designed and implemented a series of Rust libraries (called 'Crates'), for use by developers to allow them to develop IEEE 2030.5 compliant software, providing them with an abstraction for interacting with IEEE 2030.5 servers and resources.

\subsubsection{XML Serialisation \& Deserialisation Library: \- sepserde}
As we've established, resources can be communicated between clients and servers as their XML representations. This means we require the ability to serialize \& deserialize Rust data types to and from XML.
Fortunately, there already exists a popular Rust crate for this purpose, built for use in embedded communication protocols, called \texttt{YaSerde} \cite[]{YaSerde}.

However, this library does not perfectly fit IEEE 2030.5 requirements. To address this, we have forked YaSerde, and developed a crate \texttt{sepserde}, operating under a very similar interface to YaSerde, but instead producing XML representations of resources that conform to the IEEE 2030.5 specification.

\subsubsection{Common Library: \- sep2\_common}
IEEE 2030.5 resources are required for use in both clients and servers. For that reason, we've developed a common library with a Rust implementation of the IEEE 2030.5 XSD, whilst including our \texttt{sepserde} crate as to allow these resources to be serialized and deserialized to and from XML.

This common library, \texttt{sep2\_common}, can then be easily integrated and implemented in a future 2030.5 server implementation, as to avoid resources being implemented and stored differently on either. 
For the sake of modularity, and to avoid unnecessarily large binaries when compiled, this crate comes complete with compile-time flags (called Crate 'features') for each of the resource packages in IEEE 2030.5, where packages correspond to function sets.

\subsubsection{Client Library: \- sep2\_client}
The potential use cases for the IEEE 2030.5 specification are broad, as it's designed to be able to coordinate as many different types of end-user energy devices as possible.
Every implementation of a IEEE 2030.5 Client will behave differently to fit the the end-user energy devices it targets, and the model under which it is deployed. 
If a IEEE 2030.5 Client is deployed under the "Aggregated Clients Model" it will need to communicate with the end-user energy devices themselves via some undefined protocol.
If a IEEE 2030.5 Client is deployed under the "Individual/Direct Model" the very same client will be responsible for modifying the hardware of the device itself accordingly. 
Clearly, it is impossible for us as developers to implement the resulting logic for directly interacting with the electric grid.
For that reason, we have produced \texttt{sep2\_client}, a framework for developing IEEE 2030.5 Clients, regardless of the specific end-device, and regardless of the model under which it deployed.

\subsection{Open-source Software}
Our implementation will be open-sourced as it aims to address the lack of open-source IEEE 2030.5 tooling. Consequently, it's crucial that our codebase is well documented, and well tested, as to encourage and enable users to contribute modifications and fixes, and to reassure potential users of it's correctness.

As it stands, our client and common library are reasonably well-tested, with code coverage of 73\% when measured by lines of code.

Furthermore, included alongside our implementation are inline comments that are used to generate 'rustdoc' documentation from our source code. Once generated, they can be distributed as a stand-alone website documenting the features and interfaces of each crate.

\subsection{Security}
Improving the security of end-user energy devices was a major motivator behind the development of IEEE 2030 and 2030.5. 

Parallel to that measure of improving security, when considering a programming language for our implementation, we want to ensure that we are upholding the security-prioritised design of the protocol. In 2019, Microsoft attributed 70\% of all CVEs in software in the last 12 years to be caused by memory safety issues \cite[]{SecurityMemorySafety}. 

For that reason, we wish to develop our libraries in a programming language that is memory safe. A language where there are fewer attack vectors that target memory unsafety. For that reason, when implementing IEEE 2030.5 in Rust, we limit our development to the subset of Rust that provides memory safety via static analysis, 'Safe' Rust. This is opposed to using 'unsafe' Rust in our client, where raw pointers can be dereferenced, and as such the compiler is unable to provide guarantees on memory safety. When writing unsafe Rust, soundness and memory safety of unsafe code must be proven by hand.

Furthermore, in developing our implementation we must adhere to many other standards. In in the interest of security, and also correctness, we neglect implementing these standards ourselves, and instead defer to more commonly used, publicly audited, and thoroughly tested implementations of HTTP, SSL, TLS, TCP. 

An exception to this is our usage of \texttt{openssl}, which in 2023 alone has had 19 reported CVEs \cite{OpensslCVE}. In section TODO we discuss an in-progress alternative to this.

\subsection{I/O Bound Computation}
When designing our implementation, we consider that IEEE 2030.5 clients are I/O bound applications. Furthermore, with the expectation that our client library will be used to primarily develop clients operating under the client aggregation model, it is necessary that our client is able to scale alongside a large proportion of I/O bound operations.

For that reason, we require an abstraction for event-driven architecture, such that computations can be performed while waiting on I/O, without requiring any additional threads, but with the ability to leverage them as they available.

For that reason, we implement our client library using async Rust, providing us with a zero cost abstraction for asynchronous programming. When attached to an runtime, async Rust can use operating system event notifications, such as \texttt{epoll} on Linux, in order to significantly reduce the overhead of polling for new events.



\subsection{Reliability}
Per the nature of the protocol, all software used must be reliable and all expected errors are to be recovered from gracefully. Client instances must run autonomously for extended periods of time, particularly under the client aggregation model. Failure to do so could possibly lead to a denial of service for electricity.

For that reason, we leverage Rust's compile-time guarantees on the reliability of software. For example, in Rust, expected errors are to be handled at compile time. The Rust tagged union types 'Option' and 'Result' force programmers to handle error cases in order to use the output of a process that can fail. Comparatively, a language like C++ uses runtime exceptions to denote errors, such as in the standard library. C++ does not require programmers to handle these exceptions at compile-time.

As a result of Rust's type system, safe Rust also eliminates the possibility of a data race when working with multiple threads of execution, further improving the reliability of our implementation.

\subsection{Operating System}
Despite the desire to write code that is portable, our code requires a great deal of operating-system-specific functionality, and as such will need to target a single operating system. 

Of great consideration when choosing an operating system is the aforementioned 'Aggregator' model for IEEE 2030.5, where by our client would be deployed on a dedicated server, or in the cloud. In this circumstance, it's very much likely an operating system running on the Linux kernel is to be used, due to it's prevalence in server operating-systems.

Furthermore, in the event a client is being developed under the Individual/Direct model there exist very lightweight Linux based operating systems for low-spec devices. For that reason Linux based operating systems are the best candidate for our targeted operating system.

We are fortunate enough that we get, for free, a great deal of further portability by the nature of the Rust programming language, and the open-source libraries we use. Both have implementations for a wide variety of common operating systems. \cite{RustPlatforms} \cite{TokioDocs}

In the case this portability is still not sufficient, our libraries are open-source and allow users to fork our implementation and modify it for their use case, with the vast majority of our code being as portable as Rust itself.
    
\section{Assumptions}
In designing our client, we've made a set of assumptions on library user expectations, and IEEE 2030.5 Client behaviour that is not present in the spec. These assumptions have determined what functionality we have and have not implemented.

\subsection{Notification Routes}
When developing our Subscription/Notification mechanism, as part of the \texttt{sep2\_client} crate, we've made the assumption that library users will not want to a single HTTP route on their notification server that handles all incoming notifications. Instead, each notification containing a different resource would be sent to a different route. Each subscription would have a different \texttt{notificationURI} field on each subscription made.

We make this assumption as the specification makes no mention of whether a single route must be used, yet the examples in the specification show all notifications forwarded to a single URI. Under a single route, each incoming HTTP request would first need to parse the body of the request to determine how to handle it. Using multiple routes simplifies this logic, requiring only the \texttt{Host} header be inspected. 

We therefore conclude the single route usage is purely for the purpose of the example, and that in reality, client developers do not desire this functionality.

\section{Constraints}

\subsection{Generic Interface}
As we develop an implementation of the IEEE 2030.5 specification as part of this thesis, we note that we are somewhat removed from the potential use cases of our software. We have no real measure, or way to determine how one might want to use our library. 

Fortunately, we have the aforementioned existing open-source implementations to refer to. For example, the EPRI library interface was likely designed with better understanding of possible use cases, and as such, it has been appropriate to use it as a guide when designing our own interface.

Furthermore, as a general rule, we prioritise designing a highly generic interface that minimises the restrictions placed on library users as much as possible, as to support incorporating our libraries into as many differently designed Rust programs as possible, and not force any one program structure.

TODO - Rust abstractions

\subsection{EXI}
Under IEEE 2030.5, resources can be communicated between clients \& servers as their EXI representations. EXI is a binary format for XML, aiming to be more efficient (Measured by number of bytes sent for the same payload, and computation required to decode) by sacrificing human readability. As of present, there exists no Rust library for producing EXI from XML or from Rust data types, and vice-versa.
Developing a Rust EXI library fit for use in IEEE 2030.5 is a large enough of an undertaking to warrant it's own thesis, and as such, is not included in our implementation.

\subsection{DNS-SD}
The IEEE 2030.5 specification states that a connection to a server can be established by specifying a specific IP address or hostname and port. In the event cannot be provided, the specification states that DNS-SD can be used to query a local network for servers, whilst providing clients with the ability to only query for servers advertising support for specific function sets. 

TODO - Why no DNS-SD