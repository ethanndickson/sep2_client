\chapter{Adoption}\label{ch:adoption}

Further proving that the IEEE 2030.5 protocol is worth implementing is it's adoption by electric utilities and tariffs and guidelines created by government energy bodies mandating it's use.
Consequently, many implementations of the protocol exist already, with the vast majority of them proprietary, or implementing proprietary extensions of the standard.
In this section, we will examine both, and discuss how they may influence our open-source implementation. 

\section{Australian Renewable Energy Agency}


\section{California Public Utilities Commission}


\section{SunSpec}


\section{Electric Power Research Institute}
One of the more immediately relevant adoptions of the protocol is the mostly compliant implementation of a client library by EPRI in the United States.
Released under the BSD-3 license and written in the C programming language, the implementation comes in at just under twenty-thousand lines of C code.
Given that a IEEE 2030.5 client would require extension by a device manufacturer for integrating with the logic of the device itself, EPRI distributed header files as the core of the client.
For the purpose of demonstration and testing, they also bundled a client binary that can be built and executed, running as per user input.

The C codebase includes a program that parses the IEEE 2030.5 XSD schema, and converts it into C data types (structs) with documentation. This is then built with the remainder of the client library.
The implementation targets running on the Linux operating system, however, for the sake of portability, EPRI defined a set of header file interfaces that contained Linux specific API calls, such that they could be replaced for some other operating system.
These replaceable interfaces include those for networking, TCP and UDP, and event-based architecture, using the Linux \texttt{epoll} \texttt{syscall}, among others.

The IEEE 2030.5 Client implementation by EPRI states, in it's User's Manual, that it almost perfectly conforms to the IEEE 2030.5 specification according to tests written by QualityLogic.
The one exception to this is that the implementation does not support the subscription feature, as of this report.
This is particularly unusual given that the California SIP mandates the use of subscription/notification under the client aggregator model, a model of which this implementation targeted use in.
For that reason, with plans for our client to implement the subscription/notification method, we would be filling a gap in existing open-source IEEE 2030.5 software. 





