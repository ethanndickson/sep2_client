\chapter{Context}\label{ch:context}

IEEE 2030.5 is but one protocol designed for communication between end user energy devices and electric utilities. 
Although IEEE 2030.5 is a modern protocol, it is not wholly new or original, rather it is the product of historically successful protocols, designed with the same aim.
In this section we'll examine the predecessors to IEEE 2030.5, and their influence on the standard.

\section{Smart Energy Profile 1.x}
Developed by ZigBee Alliance, and published in 2008, Smart Energy Profile 1.x was a specification for an application-layer communication protocol between end-user energy devices and electric utilities. 
The specification called for the usage of the "ZigBee" communication protocol, based off the IEEE 802.15.4 specification for physical layer communication.\cite[]{ZigBeeSEP} \hfill \break
The specification was adopted by utilities worldwide, including the Southern California Edison Company, who purchased usage of the system for \$400 million USD.\cite[]{SEP1Article} \hfill \break
Unlike IEEE 2030.5 and more modern end-user energy device protocols, SEP 1.x created a HAN using ZigBee, and then used Internet Protocol to communicate data from that HAN to an electric utility.

\section{SunSpec Modbus}


\section{IEEE 2030}


\section{Producing IEEE 2030.5}



